\documentclass[twoside,11pt,ShortChapTitles]{BYUTextbook}

\usepackage{soul}
\renewcommand{\vec}[1]{\ensuremath{\mathbf{#1}}}
\usepackage{siunitx}
\sisetup{round-mode = figures,
  round-precision = 3, scientific-notation=true}
  \usepackage{marginfix}

\usepackage{mathtools}






\setcounter{chapter}{8}

\begin{document}
\chapter[Numerical Modeling II]{Numerical Modeling of a Mass-Spring System}
\section{Assignment}

{\small Complete this lab as a group and record your experience in an
organized fashion in your lab notebook. Part of the grade will be based on
neatness and organization!}

\begin{enumerate}
\item Copy, then modify your one dimensional Euler code and then
modify it to numerically model the motion of a mass-spring system with a
mass of 200.0 grams, a spring constant of 0.500 N/m, an initial displacement
from equilibrium of 15.0 cm, and an initial velocity of zero. Use a time
step of 0.01 seconds, and simulate the motion for a total of 20 seconds.
Graph your results, and comment.

\item  Repeat problem 1, but this time using a damped oscillator.
Damping is a resistive force that is opposite the direction the mass is
traveling. The resistive force has the form $F_{R}\left( n+1\right)
=bv\left( n\right) $ where$v\left( n\right) $ \ is the
mass speed and b is a constant that tells how much resistance the system has. In your modeling, use $b=0.05$  kg/s.

\item  Investigate what happens with each of the above modeling
scenarios when you increase or decrease the time step.

\item If there is time, try different masses, spring constants,
damping coefficients, and so on.

\end{enumerate}






\end{document}