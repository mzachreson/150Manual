\documentclass[twoside,11pt,ShortChapTitles]{BYUTextbook}

\usepackage{soul}
\renewcommand{\vec}[1]{\ensuremath{\mathbf{#1}}}
\usepackage{siunitx}
\sisetup{round-mode = figures,
  round-precision = 3, scientific-notation=true}
  \usepackage{marginfix}

\usepackage{mathtools}






\setcounter{chapter}{2}
\renewcommand{\chaptername}{Lab}

\begin{document}

\chapter{Measurement and Uncertainty II}

The goal of today's Lab is to measure the acceleration due to gravity, $g,$ three different ways. For each case, determine an experimental value for $g$ {\em along with its uncertainty\sidenote{Remember that if you take several measurements, you can report your value and its error as the mean and the standard deviation of the mean ($\sigma/\sqrt{N}$)}.}

Record how you find $g$ and its uncertainty for each method in your lab
notebook. Try to obtain the best value you can for each method.
\section{Finding $g$}
\subsection{Method 1: Timing a ball drop}

Using a stop watch and a tennis ball, drop the ball over a known height and
determine a value for $g.$

\subsection{Method 2: Using a pendulum}

You will learn in PH123 that a pendulum oscillates back and forth at a certain
rate. If you don't plan to take PH123, you still know that the pendulum of a
grandfather clock sets the rate at which the clock will run. The time it takes
the pendulum to go back and forth is called the \emph{period of oscillation}.
That period is given by the following equation
\[
T=2\pi\sqrt{\frac{L}{g}}
\]
where for some reason the letter $T$ stands for period, and $L$ is the length
of the pendulum (measured from the pivot point to the center of mass of the weight), and $g$ is the acceleration due to gravity. Build your
pendulum, and measure the period of oscillation using a photogate. From this
obtain a value for $g.$

\subsection{Method 3: Smart Phone Camera}

Take high speed video of a falling ball. Important things to do as you take your video:
\begin{itemize}
\item Include a meter stick or something of known length in your video, and make sure that it is about the same distance from the camera as the ball.  You do not need to have the ball fall in front of the object.
\item Try not to move the camera as you take the video
\item If you record with high speed on a cell phone, make sure that the frame rate is constant.  Many smart phones will let you start in real time, slow it down, then speed it up again.  Do not do this.
\end{itemize}




Use \emph{Logger Pro}\sidenote{Logger Pro should be installed on all lab computers.} software to analyze the video. The steps to do this in Logger Pro are outlined in the Logger Pro help under
``video analysis."

Fit a curve to your data that comes from the video. From you PH121 experience
you know that the acceleration due to gravity is constant, so we can use the
equation\sidenote{How would fitting to this equation give us a value for $g$?}
\[
y=y_{o}+v_{o}t+\frac{1}{2}at^{2}
\]
to indicate the type of curve to use for our fit. If you have trouble finding
the curve fit function in Logger Pro, or have trouble using Logger Pro, call
your instructor over.

\section{Plot Your Results}

Create a plot that shows your three different calculated values for $g$, along with errorbars

A spreadsheet program (e.g. MS Excel or LibreCalc) can graph data, and so can LoggerPro. You may know how to make a graph in one of these tools.

In this class, we are using Python, so you should try making your plot in Python.  Last week, we used matplotlib to build a histogram. The command for building a plot with errorbars is very similar.  Assuming that you've imported matplotlib as \code{plt}, the command looks like this:

\begin{Verbatim}
plt.errorbar(x,y,xerr=xerr_variable,yerr=yerr_variable,fmt='o')
\end{Verbatim}

\code{xerr} and \code{yerr} are optional commands. If you you want error bars in the x direction, and you've saved the size of your x error in the variable \code{my_x_err}, sometime after your x and y lists, you'd include the command \code{xerr=my_x_err}. If you don't have any x-error bars, leave out \code{xerr}.

  Try to make a plot of the three different values you found for $g$, with errorbars, by using the \code{errorbar} command and by borrowing and adapting parts of last week's program. The commands for labeling the axes, title, etc. for an \code{errorbar} plot are the same as the commands for a histogram.

Additionally, you can change the numbers marking the x-axis to string labels if you put these two lines of code\sidenote{This code assumes that you saved your x-values in the variable \code{x}.} somewhere after you make your plot, but before you save it:
\begin{Verbatim}
labels = ['Stopwatch','Pendulum','Video']
plt.xticks(x, labels)

\end{Verbatim}




\end{document}
