\documentclass[twoside,11pt,ShortChapTitles]{BYUTextbook}

\usepackage{soul}
\renewcommand{\vec}[1]{\ensuremath{\mathbf{#1}}}
\usepackage{siunitx}
\sisetup{round-mode = figures,
  round-precision = 3, scientific-notation=true}
  \usepackage{marginfix}

\usepackage{mathtools}

%\lstMakeShortInline[columns=fixed]|




\setcounter{chapter}{1}
\renewcommand{\chaptername}{Lab}

\begin{document}

\chapter[Statistical Representation of Data]{Communicating Results I: Statistical Representation of Data}

Complete this lab in an organized fashion in your lab notebook.

Everyone should write their own programs, but you should work together on them as a group.  Once you complete a step, stop and help your lab mates until they are caught up with you.

\subsection{Statistical Data I: How long does it take to walk?}

We will repeat one measurement, how long it takes to walk to a destination,
many times. Each person in our class will take the measurement once.

\begin{enumerate}
\item We'll start by determining a walking destination as a class. Our
destination is: \underline{\ \ \ \ \ \ \ \ \ \ \ \ \ \ \ \ \ \ \ \ }.

\item Each person in the class should get a digital timer and time their
walk to the destination and back. Walk at your normal walking speed. We will
stagger when you leave, to avoid walking in groups. While you are not
walking, you can begin working on part II.

\item When you return, record your walking time on the board to the nearest
second.

\item Record the times for all class members in a table
\item  \emph{by hand}
determine the mean walking time, the median walking time, and (if
appropriate) the modal walking time for the first 5 walkers. Determine the standard deviation of the
walking time of the first fve walkers\emph{by hand} (show all your work).

\item Using a computer: (see instructions below)
\begin{itemize}
\item Calculate the mean walking time, median walking time, and (if appropriate) the modal walking time of the class.
\item Calculate the standard deviation of the walking times using both
\begin{itemize}
\item the built in \code{numpy} function
\item the standard deviation formula coded into python: \[
\sigma _{x}=\sqrt{\sum_{i=1}^{N}\frac{\left( x_{i}-\bar{x}\right) ^{2}}{N}}
\]
\end{itemize}
\item Make a histogram of the walking times.
\end{itemize}
\end{enumerate}

\subsection{Numerical Analysis in Python}
\subsubsection{Running your first Program}
Part of this class is learning to solve physics problems with a computer.  Today, you will be using the language Python to find the median, and standard deviation of the walking data that we took.  Additionally, you will use it to make a histogram of the data.

When programming, it can be really helpful to make notes in your lab notebook about what each program does, things you learned about different functions, etc.  At the bare minimum you should include you final program, any graphs it makes, as well as where you saved it, and what name you saved it under.  That will make it easier to find in the future.

To begin writing your program, open up your favorite plain text editor.  I like notepad++ for windows or text wrangler on mac. Or, if you downloaded Python from Enthought or Anaconda you can use Canopy (Enthought) or Spyder (Anaconda).   Enter the walking data like so: (Be sure to use the class data, not this sample data.)

\begin{Verbatim}
#Our walking data
data = [34, 38, 33, 38, 38, 36, 35, 47,36, 32, 40, 40,
       45, 36, 43, 38, 48, 40, 40, 38, 43, 40, 39, 36, 46,
       34, 37, 33, 32, 34 ]
print(data)

\end{Verbatim}
The first line is called a comment.  The \code{#} tells the Python interpreter (the thing that runs Python) to ignore that line.  You should use comments to describe what your are doing in your program, that way you remember what it was later, or if anyone else has to read it they'll know what you did.  There will not be any more example comments in this tutorial.  {\em You will have to come up with your own.}

Let's look at the next set of lines:
\begin{Verbatim}
data = [34, 38, 33, 38, 38, 36, 35, 47,36, 32, 40, 40,
       45, 36, 43, 38, 48, 40, 40, 38, 43, 40, 39, 36, 46,
       34, 37, 33, 32, 34 ]

\end{Verbatim}
They load our walking data into what is called a list.  It's a way to save our data under a different name for easier access.
The \code{print(data)} command tells the computer to print out what we've saved in data.

Save your file with the extension \code{.py}. (Example: \code{myFile.py})  That tells your computer that it is a Python script.

If you are using Canopy or Spyder, you can run your script by clicking play or hitting f5.  If you aren't using one of those, open up your command line on Windows or the terminal on Linux or Mac.  Navigate the where you saved your file (ask the instructor for help if you need it) and type in the command:
\begin{Verbatim}
Python 'myFile.py'

\end{Verbatim}
but insert your file name. That tells your computer to run the Python script.  You should see your data printed out on the screen.

\subsubsection{Finding the Mean, Median, and Standard deviation}

First, remove\sidenote{You can remove code by either deleting it, or comment it out by putting a \code{\#} at the beginning of the line.  Commenting out old pieces of code can be really helpful if it's something you'd like to remember, or might use again.} the \code{print(data)} line from your program. We don't need to have the computer spit that out again and again.

Add these lines to your script:
\begin{Verbatim}
import numpy as np
dataMean=np.sum(data)/len(data)
print('Mean:  {0:.2f}'.format(dataMean))

\end{Verbatim}

The first line loads a library called \code{numpy}.  Python keeps a lot of functions in separate libraries.  Loading one is sort of like grabbing a book with the right set of instructions in it.  Now, all of the \code{numpy} functions are stored in the letters \code{np}.

The next line creates a variable called \code{dataMean}. The numpy \code{sum} command adds up all of the values in \code{data}, and \code{len()} gives you how many items are in the \code{a} list. (\code{n} is short for length).

The third line prints \code{dataMean}.  Save your program again and run it to see what happens.  Try changing the 2 in the print command line to a 3, save it, run it again, and see what has changed.

Numpy has a function that will calculate the mean for you.  Here's our script from above with one addition: \code{dataMeanNp} saves the mean of the data as calculated by numpy.
\begin{Verbatim}
import numpy as np
dataMean=np.sum(data)/len(data)
print('Mean:  {0:.2f}'.format(dataMean))
dataMeanNp=np.mean(data)
print('Numpy Mean:  {0:.2f}'.format(dataMeanNp))

\end{Verbatim}


Your assignment for this part is to {\em add} these parts to your program:
\begin{enumerate}
\item a part where the program calculates and prints the standard deviation using the formula in the reading. Hint: if you've been following the lab, \code{data-dataMeanNp} will subtract the mean from every data point stored in \code{data}.
\item a part that uses numpy to find the median and standard deviation of our walking data.  The numpy function that finds the median is \code{median} and the numpy function that finds the standard deviation is \code{std}.
\end{enumerate}
Once you find the median and standard deviation, tell your script to print the median with no decimal places and to print the standard deviation to three decimal places.

You should also include {\em comments} in your program.  Comments are notes for people to read, but that the computer will ignore.  You can start a comment with the \code{#} character.  Part of keeping a good lab notebook is printing out copies of your programs, complete with comments. Here's the example program, all in one place, with good comments added:
\begin{Verbatim}
#Load the class walking times into the variable "data"
data = [34, 38, 33, 38, 38, 36, 35, 47,36, 32, 40, 40,
       45, 36, 43, 38, 48, 40, 40, 38, 43, 40, 39, 36, 46,
       34, 37, 33, 32, 34 ]

#Importing the numpy library for easy calculations
import numpy as np

#Calculate the mean of data using the mean formula
dataMean=np.sum(data)/len(data)
#Print the mean as a float with two decimal places
print('Mean:  {0:.2f}'.format(dataMean))

#Calculate the mean of the data using numpy's mean function
#If I did dataMean correctly, dataMeanNp should give the same result
dataMeanNp=np.mean(data)
#Print the mean calculated from numpy's function.
print('Numpy Mean:  {0:.2f}'.format(dataMeanNp))

\end{Verbatim}

\subsubsection{Making a histogram}

Adding the following code to your script to make the histogram:

\begin{Verbatim}
import matplotlib.pyplot as plt
plt.hist(data, 20, normed=0, facecolor='green', alpha=0.75)

plt.xlabel('My x axis Label')
plt.ylabel('My y axis label')
plt.title('My Title')
plt.savefig('myPlot.pdf')

\end{Verbatim}

This script will make a histogram with 20 bins and save it to the file \code{myPlot.pdf}.  In your program you should:
\begin{itemize}
\item Change the number of bins to something more appropriate for your data. If your fullest bin only has one or two items in it, you have way too man bins.  If everything fits into three or four bins, you have too few.
\item Fix the plot title and axis labels to match what {\em you} are plotting.
\item Add appropriate comments. If you have enough comments, it can be very helpful to refer back to this program in future weeks.  If you don't have enough, you will forget what this program does, and it won't be helpful to you.
\end{itemize}

Print out your completed program and histogram and add them to your lab notebook.
\end{document}
