\documentclass[twoside,11pt,ShortChapTitles]{BYUTextbook}

\usepackage{soul}
\renewcommand{\vec}[1]{\ensuremath{\mathbf{#1}}}
\usepackage{siunitx}
\sisetup{round-mode = figures,
  round-precision = 3, scientific-notation=true}
  \usepackage{marginfix}

\usepackage{mathtools}






\setcounter{chapter}{3}
\renewcommand{\chaptername}{Lab}
\begin{document}


\chapter[Experimental Design I]{Experimental Design I: Harmonic Oscillators (masses and springs)\label{Experimental Design}}


This week's assignment follows the experimental design process outlined in the introduction to the lab.  We are trying to determine whether or not
I have mostly designed this experiment for you. So this week I want you to
identify the design parts and put them in our design process order. For our
next design lab, you will have to design the experiment yourself. This week's
lab is to get familiar with the process. Perform this experiment as a group.

\section{First Part: Data Collection}

\begin{enumerate}
\item Our system will be the mass-spring system and it's hanger. Obtain a set
of weights, a spring, a weight hanger, a stand, and a stopwatch. Attach the
spring to the stand, and the weight hanger to the spring. Determine the inputs
to this mass-spring system that may affect the output quantity of interest
(the period of oscillation). Determine whether each of these inputs will
affect the period of oscillation. If so, explain how you will control for that
input. If not, give justification for why you can ignore that input.

\item Build a mathematical model beginning with the suggestion you got from
reading Hook's work (above).

\item Determine how you will measure the period of oscillation. Remember that
you want to minimize the amount of uncertainty in your measurement. Techniques
we have learned in previous labs may help. Record your method. You should plan
any graphs you will make and in general plan how you will report your data and
whether or not your experiment is successful.

\item Discuss how you might go about making your equation look linear by a
proper substitution of variables. Explain why this might be useful.

\item Select a range of variables. (e.g. $m=$ $20
\operatorname{g}
$, $30
\operatorname{g}
$, $40
\operatorname{g}
$, $50
\operatorname{g}
$, and $100
\operatorname{g}
$). Don't use $80
\operatorname{g}
$ because I want to reserve this value for a special purpose below. Stop at
about $100
\operatorname{g}
.$

\item Plan your procedure and record your plan in your lab notebook.

\item Perform the experiment. Your plan probably includes determining the
period of oscillation for masses that you have selected. Be sure to record any
measurement uncertainty. Make the graphs of your data that you planned
including the appropriate error bars and a best fit line (See second part). Attach the graphs in your lab notebook.
Record what you do and highlight any deviations from your planned procedure.
Record your data or your data file name and location. Show your analysis and
give your results. Draw conclusions. We will check these conclusions in the
third part. But state whether you believe that $T\propto\sqrt{m}$
\end{enumerate}

\section{Second Part: fitting a line}
In order to fit a line to your data, you'll need to teach Python how to do a linear least squares fit using the equations and the example function in the lab introduction.  You will be doing many linear least squares fits, so it is in your best interest to create a least squares function in its own file that you can use in the future.  That way you don't have to copy and paste it over and over again. These steps will teach you how.
\begin{enumerate}
\item Linearize your equation.  Remember, linear least squares fitting only works on lines.
\item Type out the example program given in this lab's intro, and save it in the folder where you have been saving your other programs from lab.  Make sure that you match the indentation from the example.  Python uses indentation to tell where functions end.  Here's a quick example:
\begin{Verbatim}
def line(x,m,b):
    return m*x+b
\end{Verbatim}

Notice the difference in indentation between the two lines. The "def" command tells Python that you want to create your own function.  In this case, the function is called \code{line} and takes inputs \code{x}\sidenote{This function assumes that \code{x} is a numpy array.  If it is a regular list, the multiplication won't work the way you expect it to.},\code{m}, and \code{b}.  The return command tells Python what you want to get out of your function.  You could define a function this way and get the same result:

\begin{Verbatim}
def line(x,m,b):
    y=m*x+b
    return y
\end{Verbatim}

But you will get an error if you enter this:
\begin{Verbatim}
def line(x,m,b):
    y=m*x+b
return y
\end{Verbatim}
Indentation is very important in Python.  When you define a function, everything that is indented below it is counted as part of that function.  Not indenting the return statement tells Python that you've ended your function, and it won't know what to return.
Here's one more example:
\begin{Verbatim}
def line(x,m,b):
    y=m*x+b
    return y

y=line(5,2,1)
\end{Verbatim}
The previous piece of code tells Python what we want our function to be.  Then,  the very last line (notice that it is not indented) tells Python that we want to use our function, and that we want \code{x=5}, \code{m=2}, and \code{b=1}.  Python will perform the operation and save the number 11 (the result of 5*2+1) into the variable \code{y}.  Python will also treat any variables inside of functions as separate from the ones outside, meaning the \code{y} in our function is independent of the \code{y} in the last line. Once you define a function, you can use it over and over again in your program.

\item Test your function.  Put these lines of code {\em after} your function\sidenote{The \code{if __name__ == "__main__":} line tells Python to only run this part of the program when you are running this particular file.  If you leave it out, Python will run this test whenever you load your function into another file.}:
\begin{Verbatim}
if __name__ == "__main__":
    xdata=[1,2,3,4,5,6]
    ydata=[1,2,3,4,5,6]

    #Run linear least squares fit on the data
    slope, intercept = linear_least_squares(xdata,ydata)

    #Print out the test values
    print('Slope: {}'.format(slope))
    print('Intercept: {}'.format(intercept))

\end{Verbatim}
If your program is working properly, it will print:
\begin{Verbatim}
Slope: 1.0
Intercept: 0.0
\end{Verbatim}

\item It is also very important to be able to calculate error.  Modify the \code{linear_least_squares} function so that it also returns the error in the slope ($\sigma_m$) and intercept ($\sigma_b$).  As a reminder, these equations calculate the error:
\[\sigma_m=\frac{\sigma_y}{\sqrt{N\left(\left<x^2\right>-\left<x\right>^2\right)}}\]
\[\sigma_b=\sigma_y\sqrt{\frac{\left<x^2\right>}{N\left(\left<x^2\right>-\left<x\right>^2\right)}}\]
where
\[\sigma_y=\sqrt{\frac{1}{N-2}\sum_i^N \left(y_i-b-mx_i\right)^2}\]

The error in both the slope and intercept should be zero if your fit is working properly.
\end{enumerate}

Now start building a script to fit this week's data.
\begin{enumerate}
\item Open up a new Python script and save it in the same directory as your least squares file.  Put this command at the top of your file: (For this example, the file with the least squares fitting functions in it was saved with the name \code{linear_least_squares.py}, change the name to match accordingly.)
\begin{Verbatim}
import linear_least_squares
\end{Verbatim}
This line tells Python to load all of the functions in \code{linear_least_squares.py}.  You can now use your least squares fitting function in this program.

\item Load your data into your program. Doing math with data sets is easier in Python if you use the numpy library.  It can be very helpful to save our data like this\sidenote{It would be good to give your data more descriptive names than \code{x_data} and \code{y_data}, and add comments.}:
\begin{Verbatim}
import numpy as np
x_data=np.asarray([1,2,3,4,5])
y_data=np.asarray([2.1, 3.9, 5.8, 8.4, 11])
\end{Verbatim}
As an experiment, tell Python to \code{print(x_data*y_data)} and see what happens.    Numpy arrays can only have numbers in them, and so multiplying two numpy arrays will multiply each item in the list individually.  Since Python lists can have anything in them (numbers, words, other lists) Python doesn't have a built in way to do math on entire lists without doing quite a bit more work.  That's mostly because it doesn't make any sense to say \code{5*'hello'}.

\item Do a least squares fit to your data using your fitting functions.

\item Check your fit equation by using your fitted slope and intercept to
calculate a new set of \code{y} values (\code{yfit=m*x+b}) and then plotting them. (Data points with errorbars, as well as the fit line) The \code{matplotlib.pyplot} command for plotting a regular line is \code{plot(x_points,y_points)}.


\end{enumerate}

\section{Third Part: Interpolation and Extrapolation}

We would like to test the equation or  ``law" you developed in the last part. We will use the equation to predict periods
for masses you have not yet used.

\begin{enumerate}
\item \textbf{By interpolation, predict the period of oscillation for an }$80
\operatorname{g}
$\textbf{\ mass.} Record your methods and results. Interpolation means to
predict an output value (in this case, a period) for an input value that falls
within the range of the input values you have used in your measurements. If
you measured periods for $20
\operatorname{g}
$, $30
\operatorname{g}
$, $40
\operatorname{g}
$, $50
\operatorname{g}
$, and $100
\operatorname{g}
,$ then $80
\operatorname{g}
$ is within this range. Using the curve fit equation generated by the data we
measured, we can plug in $80
\operatorname{g}
$ and predict the period for our spring with an $80
\operatorname{g}
$ mass. This is interpolation. This will test our model to see if it works for
new inputs. If it does not, our model is probably not good.

\item \textbf{By extrapolation, predict the period of oscillation for a }$300
\operatorname{g}
$\textbf{\ mass.} Record your methods and results. Extrapolation means to
predict an output value (in this case, a period) for an input value that falls
outside the range of the input values you have previously measured. If you
measured periods for $20
\operatorname{g}
$, $30
\operatorname{g}
$, $40
\operatorname{g}
$, $50
\operatorname{g}
$, and $100
\operatorname{g}
,$ then $300
\operatorname{g}
$ is outside this range. Using the curve fit equation generated by the data we
measured, we can input $300
\operatorname{g}
$ and predict the period for our spring with an $300
\operatorname{g}
$ mass. Extrapolation is more risky. The conditions of our experiment might
change outside our range (think, in a limiting case, we could break the
spring, and get an infinite period!). But if things are done carefully, this
is also a test of the validity of our model.

\item Measure the period of oscillation for the $80
\operatorname{g}
$ and $300
\operatorname{g}
$ masses. Be sure to account for all uncertainties. Compare your measurements
with your predictions, and comment on the level of agreement.

\item Now that we have tested our mathematical model for the relationship
between period and mass for a mass-spring system, you can report it. Determine
values for your constants, including uncertainties . Record your methods and results.
\end{enumerate}

\subsection{Third Part: Further Discussion}

\begin{enumerate}
\item An often useful tool, especially when your data is not naturally linear,
is to plot it on a logarithmic scale. Create such a graph using Python and attach it to your lab notebook.  The matplotlib function is \code{semilogy(x\_data,y\_data)}. Comment on what you see.

\item Don't forget to make good comments on what you did and how you did it in
your lab book.
\end{enumerate}

\end{document}
