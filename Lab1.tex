\documentclass[twoside,11pt,ShortChapTitles]{BYUTextbook}

\usepackage{soul}
\renewcommand{\vec}[1]{\ensuremath{\mathbf{#1}}}
\usepackage{siunitx}
\sisetup{round-mode = figures,
  round-precision = 3, scientific-notation=true}
  \usepackage{marginfix}

\usepackage{mathtools}

\renewcommand{\chaptername}{Lab}

\setcounter{chapter}{0}

\begin{document}

\chapter{Measurement and Uncertainty I\label{Measurement and Uncertainty 1}}

%\setcounter{page}{1}

\section{Lab Notebooks}

Hopefully you noticed that a lab notebook is required for this class. The lab
notebook is designed to be a record of what you did. If you had to repeat
today's experiment five years from now, could you do it based on what you
write today?

At most professional labs and major engineering companies your lab notebook is
considered the property of the company or organization and will stand as a legal document. It is the proof that
you did the experiment that you say you did, and that you got the results you
say you got. It has to be readable and understandable to someone who did not
participate in the lab with you. This is a pretty tall order.

{\em You should write  in your lab notebook as you go}, not leave it until the end of the day\sidenote{A lab notebook is not a lab report.  You just need to take well organized notes on what you did and what you found.}.  It will be much easier, and will take you less time as you go.   To help you plan your entries, here are the criteria I will use to grade your lab book:

\begin{itemize}
\item Describing the goal for the work

\begin{itemize}
\item Usually this takes the form of a physical law we will test.
\end{itemize}

\item Give predictive equations and uncertainties for the predictions based on
the physical law.

\begin{itemize}
\item This usually involves forming a mathematical model. You should record
any assumptions that went into the model (e.g. no air resistance, point
sources, massless ropes, etc.).

\begin{itemize}
\item In lab today we will find the volume of the room. Your mathematical
model will likely be $V=\ell\times w\times h.$ The mathematical model is not
necessarily something complicated, but the reader needs to know how you are
doing your calculations.
\end{itemize}
\end{itemize}

\item Give your procedure

\begin{itemize}
\item Recording what you really did (not the lab instructions), tell what
changes you make in your procedure as you make them.

\item Record as you do the work.

\item Record the equipment used and settings, values, etc. for that equipment
(see next item).

\item Did you learn how to use any new equipment? What did you learn that you
want to recall later (say, when taking the final, or when you are a
professional and need to use a similar piece of equipment five years from now).
\end{itemize}

\item Record the data you used. The data are all the measurements you took
plus your best estimate of the uncertainties in the measurements. Record any
values you got from tables or published sources (or from your professor) and
state where you got these values. You don't always want to write down all the
data you use. If you have a large set of values, you can place them in a file,
and then record the file name and location in your lab notebook. Make sure
this is a file location that does not change (emailing the data to yourself is
not a good plan).

\item Give a record of the analysis you performed. You should have given some
idea of how you got your predictive equation. Now, what did you do to get the
data through the equation? Were there any extra calculations? Did you obtain a
set of  ``truth data" (data from tables or
published sources, or from an alternate experiment) for your experiment? If
so, did you do any calculations, have any uncertainty, etc. associated with
the truth values?

\item Give a brief statement of your results and their associated uncertainties.

\item Draw conclusions

\begin{itemize}
\item Do your results support the theory? Why or why not? What else did you
learn along the way that you want to record.

\item This is where we may compare the percent error to our relative uncertainty.
\end{itemize}
\end{itemize}



\section{Assignment: Practice with Measurement and Uncertainty calculations.}

\subsection{Part 1 Percent Error: Mass of a Cylinder--the hard way}

\begin{itemize}
\item Given the density of a metal cylinder, use this density to determine the
mass the cylinder.

\begin{itemize}
\item You cannot directly measure the mass of the cylinder, You will be
provided a mass of the cylinder by your instructor to compare with your
calculated value.

\item Report your method for obtaining the mass of the cylinder in your lab
notebook (not just your result, but tell yourself in your notebook \emph{how
you got your result}).

\item Report the following results:\sidenote{Notice that the uncertainty in the calculated volume isn't in this list.  You don't have to find it yet.} 1) Density of the cylinder, 2) Predicted
Mass of the cylinder, 3) Actual Mass of the cylinder. Comment on the accuracy
and precision of your measurement.

\item Resources: You may use any equipment or other resources found in the lab
or on the internet
\end{itemize}
\end{itemize}

\subsection{Part 2 Combining Uncertainty: Volume of the room}

\begin{itemize}
\item Determine the volume of this room, {\em including uncertainties}. Describe
your method fully in your lab notebook, including which measuring instruments
you used and why, and the uncertainty\sidenote{Even though the uncertainty equations are given below, you should try to show where they came from in your lab notebook.  If you need help doing this, be sure to ask for it.} in each of your measurements.

\item The {\em absolute} uncertainty in the volume is
\[
\delta V=\left(\frac{\delta L}{L}+\frac{\delta H}{H}+\frac{\delta W}{W}\right)V
\]
from our algebraic method multiplication rule or
\[
\left(\frac{\delta V}{V}\right)^2=\left(\frac{\delta L}{L}\right)^2+\left(\frac{\delta H}{H}\right)^2+\left(\frac{\delta W}{W}\right)^2
\]

\item Compare your answers with those from your neighboring research
institutions at the other tables. Are your answers the same to within the
values of your uncertainty? If not, explain why they aren't.
\end{itemize}

\subsection{Part 3: Tie to Experimentation}

\begin{itemize}
\item We will learn in this class that you should understand the uncertainties
in our measuring devices \emph{before} you start performing an experiment.
From what you have experienced so far today, why do you think this is so?
\end{itemize}



\subsection{Part 4 Combining Uncertainty: Determine the Volume of a Stack of
Paper}

\begin{itemize}
\item Determine the volume of $20$ pieces of paper (you can use more, but if
you do, replace the number $20$ with your actual number in the equation below).

\item Determine the uncertainty in your measurement.

\item Use your measurement to find the volume of one sheet of paper by
dividing. Also determine the uncertainty in your calculation. This should be
something like
\[
\delta V_{1}=\frac{\delta V_{20}}{20}
\]
explain what this means in your lab notebook.

\item Now measure the volume of one piece of paper directly using instruments
(I might recommend a micrometer--ask if you have not used one before).

\item How do your measurements compare?

\item Which one is more accurate? Which is more precise? Why?
\end{itemize}


\end{document}









%EXCISED MATERIAL==================================================

We can see that if you have $z=x+y$ then you just add the uncertainties
$\delta z=\delta x+\delta y.$ Of course $z$, $x,$ and $y$ stand for any
variable. But how do we know this is true? Here are the details of how the
algebraic method works

\subsubsection{Multiplication}

To multiply two measurements, say\[
m_{measured}=m_{N}\pm\delta m
\]\[
v_{measured}=v_{N}\pm\delta v
\]


We could write these as\[
m_{measured}=m_{N}\left(  1\pm\frac{\delta m}{\left\vert m_{N}\right\vert
}\right)
\]\[
v_{measured}=v_{N}\left(  1\pm\frac{\delta v}{\left\vert v_{N}\right\vert
}\right)
\]


This gives us the measurement in terms of the fractional uncertainties. If we
wish to compute
\[
p=mv
\]
we use
\[
p_{N}=m_{N}v_{N}
\]
but what is the uncertainty in $p?$

The largest value of $p$ is given by\[
p_{\text{large}}=m_{\mathbf{N}}v_{\mathbf{N}}\left(  1+\frac{\delta
m}{\left\vert m_{N}\right\vert }\right)  \left(  1+\frac{\delta v}{\left\vert
v_{\mathbf{N}}\right\vert }\right)
\]
which can be written as\[
p_{\text{large}}=m_{\mathbf{N}}v_{\mathbf{N}}\left(  1+\frac{\delta
m}{\left\vert m_{\mathbf{N}}\right\vert }+\frac{\delta v}{\left\vert
v_{\mathbf{N}}\right\vert }+\frac{\delta v}{\left\vert v_{\mathbf{N}}\right\vert }\frac{\delta m}{\left\vert m_{\mathbf{N}}\right\vert }\right)
\]
We reason that fractional uncertainties should be small, so products of
fractional uncertainties should be very small. We will ignore the very small
term\[
\frac{\delta m}{\left\vert v_{\mathbf{N}}\right\vert }\frac{\delta
m}{\left\vert m_{\mathbf{N}}\right\vert }
\]
so we have\[
p_{\text{large}}=m_{\mathbf{N}}v_{\mathbf{N}}\left(  1+\frac{\delta
m}{\left\vert m_{\mathbf{N}}\right\vert }+\frac{\delta v}{\left\vert
v_{\mathbf{N}}\right\vert }\right)
\]
The smallest value of $p$ is likewise\[
p_{\text{small}}=m_{\mathbf{N}}v_{\mathbf{N}}\left(  1-\frac{\delta
m}{\left\vert m_{\mathbf{N}}\right\vert }-\frac{\delta v}{\left\vert
v_{\mathbf{N}}\right\vert }\right)
\]
In each case we have $m_{\mathbf{N}}v_{\mathbf{N}}$ and then either plus or
minus the term $\frac{\delta m}{\left\vert m_{\mathbf{N}}\right\vert }+\frac{\delta v}{\left\vert v_{\mathbf{N}}\right\vert }$so we have, for our
calculated value of $p$\[
p_{\text{calculated}}=m_{\mathbf{N}}v_{\mathbf{N}}\left(  1\pm\left(
\frac{\delta m}{\left\vert m_{\mathbf{N}}\right\vert }+\frac{\delta
v}{\left\vert v_{\mathbf{N}}\right\vert }\right)  \right)
\]
which we must be able to write as a nominal value and an uncertainty in $p$\[
p_{\text{calculated}}=p_{\mathbf{N}}\left(  1\pm\frac{\delta p}{\left\vert
p_{\mathbf{N}}\right\vert }\right)
\]
Comparing the previous two equations we can see that we must have\[
\frac{\delta p}{\left\vert p_{\mathbf{N}}\right\vert }=\left(  \frac{\delta
m}{m_{\mathbf{N}}}+\frac{\delta v}{v_{\mathbf{N}}}\right)
\]


Then when we multiple measured quantities, we add fractional uncertainties.
This is our algebraic rule for multiplication.

Don't forget, that we need to report $\delta p,$ so to find $\delta p$ we
take\[
\delta p=\frac{\delta p}{\left\vert p_{\mathbf{N}}\right\vert }p_{\mathbf{N}}
\]


\subsubsection{Division}

Let's start again with two measured values $x$ and $y$ with the form\[
x=x_{\mathbf{N}}\left(  1\pm\frac{\delta x}{\left\vert x_{\mathbf{N}}\right\vert }\right)
\]\[
y=y_{\mathbf{N}}\left(  1\pm\frac{\delta y}{\left\vert y_{\mathbf{N}}\right\vert }\right)
\]
We can find the quotient
\[
q=q_{\mathbf{N}}\pm\delta q
\]
as we did for multiplication\[
q=\frac{x_{\mathbf{N}}\left(  1\pm\frac{\delta x}{\left\vert x_{\mathbf{N}}\right\vert }\right)  }{y_{\mathbf{N}}\left(  1\pm\frac{\delta y}{\left\vert
y_{\mathbf{N}}\right\vert }\right)  }
\]
Again find the maximum quotient\[
q_{\text{large}}=\frac{x_{\mathbf{N}}\left(  1+\frac{\delta x}{\left\vert
x_{\mathbf{N}}\right\vert }\right)  }{y_{\mathbf{N}}\left(  1-\frac{\delta
y}{\left\vert y_{\mathbf{N}}\right\vert }\right)  }
\]
and we will play a mathematical trick, we will multiple both top and bottom by
$\left(  1+\frac{\delta y}{\left\vert y_{\mathbf{N}}\right\vert }\right)
,$ since
\[
\frac{\left(  1+\frac{\delta y}{\left\vert y_{\mathbf{N}}\right\vert }\right)
}{\left(  1+\frac{\delta y}{\left\vert y_{\mathbf{N}}\right\vert }\right)
}=1
\]
this will not change our value for $q_{\text{large}}$\[
q_{\text{large}}=\frac{x_{\mathbf{N}}\left(  1+\frac{\delta x}{\left\vert
x_{\mathbf{N}}\right\vert }\right)  }{y_{\mathbf{N}}\left(  1-\frac{\delta
y}{\left\vert y_{\mathbf{N}}\right\vert }\right)  }\frac{\left(
1+\frac{\delta y}{\left\vert y_{\mathbf{N}}\right\vert }\right)  }{\left(
1+\frac{\delta y}{\left\vert y_{\mathbf{N}}\right\vert }\right)  }
\]
The denominator is\[
y_{\mathbf{N}}\left(  1-\frac{\delta y}{\left\vert y_{\mathbf{N}}\right\vert
}\right)  \left(  1+\frac{\delta y}{\left\vert y_{\mathbf{N}}\right\vert
}\right)
\]
We can perform the multiplication to get\begin{align*}
& y_{\mathbf{N}}\left(  1-\frac{\delta y}{\left\vert y_{\mathbf{N}}\right\vert
}+\frac{\delta y}{\left\vert y_{\mathbf{N}}\right\vert }-\frac{\delta
y}{\left\vert y_{\mathbf{N}}\right\vert }\frac{\delta y}{\left\vert
y_{\mathbf{N}}\right\vert }\right) \\
& =y_{\mathbf{N}}\left(  1-\frac{\delta y}{\left\vert y_{\mathbf{N}}\right\vert }\frac{\delta y}{\left\vert y_{\mathbf{N}}\right\vert }\right)
\end{align*}
and we can perform the multiplication in the numerator
\[
x_{\mathbf{N}}\left(  1+\frac{\delta x}{\left\vert x_{\mathbf{N}}\right\vert
}+\frac{\delta y}{\left\vert y_{\mathbf{N}}\right\vert }+\frac{\delta
x}{\left\vert x_{\mathbf{N}}\right\vert }\frac{\delta y}{\left\vert
y_{\mathbf{N}}\right\vert }\right)
\]
so\[
q_{\text{large}}=\frac{x_{\mathbf{N}}\left(  1+\frac{\delta x}{\left\vert
x_{\mathbf{N}}\right\vert }+\frac{\delta y}{\left\vert y_{\mathbf{N}}\right\vert }+\frac{\delta x}{\left\vert x_{\mathbf{N}}\right\vert }\frac{\delta y}{\left\vert y_{\mathbf{N}}\right\vert }\right)  }{y_{\mathbf{N}}\left(  1-\frac{\delta y}{\left\vert y_{\mathbf{N}}\right\vert
}\frac{\delta y}{\left\vert y_{\mathbf{N}}\right\vert }\right)  }
\]
The term\[
\frac{\delta y}{\left\vert y_{\mathbf{N}}\right\vert }\frac{\delta
y}{\left\vert y_{\mathbf{N}}\right\vert }
\]
is very small compared to $1$ (if $\delta y/\left\vert y_{\mathbf{N}}\right\vert $ is a small number, as we assume, then $\delta y^{2}/\left\vert
y_{\mathbf{N}}\right\vert ^{2}$ will be very tiny) so we will drop it from
our calculations (we are calculating uncertainty, the fifth decimal place in
the uncertainty is very uncertain!). We can do this as well with the term\[
\frac{\delta x}{\left\vert x_{\mathbf{N}}\right\vert }\frac{\delta
y}{\left\vert y_{\mathbf{N}}\right\vert }
\]
like we did with the multiplication case. Then\begin{align*}
q_{\text{large}}  & =\frac{x_{\mathbf{N}}\left(  1+\frac{\delta x}{\left\vert
x_{\mathbf{N}}\right\vert }+\frac{\delta y}{\left\vert y_{\mathbf{N}}\right\vert }\right)  }{y_{\mathbf{N}}\left(  1\right)  }\\
& =\frac{x_{\mathbf{N}}}{y_{\mathbf{N}}}\left(  1+\frac{\delta x}{\left\vert
x_{\mathbf{N}}\right\vert }+\frac{\delta y}{\left\vert y_{\mathbf{N}}\right\vert }\right)
\end{align*}
We could go through this again for $q_{\text{small}}$ and we would find
\begin{align*}
q_{\text{small}}  & =\frac{x_{\mathbf{N}}\left(  1-\left(  \frac{\delta
x}{\left\vert x_{\mathbf{N}}\right\vert }+\frac{\delta y}{\left\vert
y_{\mathbf{N}}\right\vert }\right)  \right)  }{y_{\mathbf{N}}\left(  1\right)
}\\
& =\frac{x_{\mathbf{N}}}{y_{\mathbf{N}}}\left(  1-\left(  \frac{\delta
x}{\left\vert x_{\mathbf{N}}\right\vert }+\frac{\delta y}{\left\vert
y_{\mathbf{N}}\right\vert }\right)  \right)
\end{align*}


We can then write\begin{align*}
q_{\text{calculated}}  & =\frac{x_{\mathbf{N}}}{y_{\mathbf{N}}}\left(
1\pm\left(  \frac{\delta x}{\left\vert x_{\mathbf{N}}\right\vert }+\frac{\delta y}{\left\vert y_{\mathbf{N}}\right\vert }\right)  \right) \\
& =q_{\mathbf{N}}\left(  1\pm\frac{\delta q}{\left\vert q\right\vert }\right)
\end{align*}
where
\[
q_{\mathbf{N}}=\frac{x_{\mathbf{N}}}{y_{\mathbf{N}}}
\]
and\[
\frac{\delta q}{\left\vert q\right\vert }=\frac{\delta x}{\left\vert
x_{\mathbf{N}}\right\vert }+\frac{\delta y}{\left\vert y_{\mathbf{N}}\right\vert }
\]
just like the formula for multiplication! The rule is when we divide measured
quantities, we add fractional uncertainties

\subsubsection{Addition}

Let's take two measurements\[
x_{measured}=x_{\mathbf{N}}\pm\delta x
\]\[
y_{measured}=y_{\mathbf{N}}\pm\delta y
\]
and add them to get the largest value of the sum, $z$\begin{align*}
z_{\text{large}}  & =x_{measured}+y_{measured}\\
& =x_{\mathbf{N}}+y_{\mathbf{N}}+\delta x+\delta y
\end{align*}
We can also find the smallest value for $z$\begin{align*}
z_{\text{small}}  & =x_{measured}+y_{measured}\\
& =x_{\mathbf{N}}+y_{\mathbf{N}}-\left(  \delta x+\delta y\right)
\end{align*}
so if we write this as $z_{\mathbf{N}}\pm\delta z$ we have\[
z_{\mathbf{N}}\pm\delta z=x_{\mathbf{N}}+y_{\mathbf{N}}\pm\left(  \delta
x+\delta y\right)
\]
and we can identify
\begin{align*}
z_{\mathbf{N}}  & =x_{\mathbf{N}}+y_{\mathbf{N}}\\
\delta z  & =\left(  \delta x+\delta y\right)
\end{align*}


The rule is that when we add measurements we add their uncertainties.

\subsubsection{Subtraction}

Let's take two measurements\[
x_{measured}=x_{\mathbf{N}}\pm\delta x
\]\[
y_{measured}=y_{\mathbf{N}}\pm\delta y
\]
and add them to get the largest value of the sum, $z$\begin{align*}
z_{\text{large}}  & =x_{measured}-y_{measured}\\
& =\left(  x_{\mathbf{N}}+\delta x\right)  -\left(  y_{\mathbf{N}}-\delta
y\right)
\end{align*}
We can also find the smallest value for $z$\begin{align*}
z_{\text{small}}  & =x_{measured}-y_{measured}\\
& =\left(  x_{\mathbf{N}}-\delta x\right)  -\left(  y_{\mathbf{N}}+\delta
y\right)
\end{align*}
so if we write this as $z_{\mathbf{N}}\pm\delta z$ we have\[
z_{\mathbf{N}}\pm\delta z=x_{\mathbf{N}}-y_{\mathbf{N}}\pm\left(  \delta
x+\delta y\right)
\]
and we can identify
\begin{align*}
z_{\mathbf{N}}  & =x_{\mathbf{N}}-y_{\mathbf{N}}\\
\delta z  & =\left(  \delta x+\delta y\right)
\end{align*}


The rule is that when we subtract measurements, we add their uncertainties
just as we found for addition.

\subsubsection{Measured quantity times an exact number}

Suppose we want the circumference of a circle. We measure the radius to be\[
r_{measured}=r_{\mathbf{N}}\pm\delta r
\]
and we calculate
\[
C_{calculated}=2\pi r_{measured}
\]
how do we determine the uncertainty?

There is not uncertainty in the $2$ nor in the $\pi.$ So we multiply\begin{align*}
C_{calc}  & =2\pi\left(  r_{\mathbf{N}}\pm\delta r\right) \\
& =2\pi r_{\mathbf{N}}\pm2\pi\delta r
\end{align*}
which gives
\[
C_{\mathbf{N}}\pm\delta C=2\pi r_{\mathbf{N}}\pm2\pi\delta r
\]
and we identify
\[
C_{\mathbf{N}}=2\pi r_{\mathbf{N}}
\]
and
\[
\delta C=2\pi\left\vert \delta r\right\vert
\]


In general, if we have
\[
q=Bx
\]
where $B$ is a constant, then we should expect\[
\delta q=\left\vert B\right\vert \delta x
\]


\paragraph{Powers}

For a power, we really are just multiplying\[
y=x^{2}=x\times x
\]
so, taking the fractional uncertainty rule for multiplication,\begin{align*}
\frac{\delta y}{\left\vert y\right\vert }  & =\frac{\delta x}{\left\vert
x\right\vert }+\frac{\delta x}{\left\vert x\right\vert }\\
& =2\frac{\delta x}{\left\vert x\right\vert }\end{align*}


In general, if
\[
y=x^{n}
\]
then
\[
\frac{\delta y}{\left\vert y\right\vert }=n\frac{\delta x}{\left\vert
x\right\vert }
\]

