\documentclass[twoside,11pt,ShortChapTitles]{BYUTextbook}

\usepackage{soul}
\renewcommand{\vec}[1]{\ensuremath{\mathbf{#1}}}
\usepackage{siunitx}
\sisetup{round-mode = figures,
  round-precision = 3, scientific-notation=true}
  \usepackage{marginfix}

\usepackage{mathtools}

\usepackage{listings}

\usepackage{color}

\definecolor{dkgreen}{rgb}{0,0.6,0}
\definecolor{gray}{rgb}{0.5,0.5,0.5}
\definecolor{mauve}{rgb}{0.58,0,0.82}

\lstset{frame=tb,
  language=Python,
  aboveskip=3mm,
  belowskip=3mm,
  showstringspaces=false,
  columns=flexible,
  basicstyle={\small\ttfamily},
  numbers=none,
  numberstyle=\tiny\color{gray},
  keywordstyle=\color{blue},
  commentstyle=\color{dkgreen},
  stringstyle=\color{mauve},
  breaklines=true,
  breakatwhitespace=true,
  tabsize=3, upquote=true}

\lstMakeShortInline[columns=fixed]|
\setcounter{chapter}{6}

\begin{document}



\chapter[Numerical modeling I]{Numerical modeling of Projectile motion}


\section{Assignment}

{\small Because we are learning a major new skill. we will take three weeks to
complete the experiment we are starting today. This will affect your lab
notebook. Complete each part of the lab each week in an organized fashion in
your lab notebook. Make graceful end-of-class entries so you can start up
again the following week. As always, part of the grade will be based on
neatness and organization!}

\begin{enumerate}
\item {\small Create a Python script that will numerically model the motion of
a spherical projectile shot straight up and then falling back down using
Euler's method. Assume there is no air resistance. As input quantities you
should provide the following:}

\begin{itemize}
\item {\small The initial }$y${\small \ position of the projectile } $y=70.00${\small \ meters}

\item {\small The initial speed of the projectile. }$30.0 \operatorname{m} / \operatorname{s} $

\item {\small The time step size.}$0.1${\small \ seconds}

\item {\small The acceleration due to gravity }$9.8 \operatorname{m} / \operatorname{s} ^{2}$
\end{itemize}

{\small Make these quantities variables so you can easily change values and
recalculate. The program in the precious section will walk you through this part. Save your script and record where you saved it.
Include a scatter plot of }${\small y}${\small \ vs }${\small t}${\small \ in
your lab notebook.}

\item {\small Copy and then modify your Python script to numerically model the
motion of a spherical projectile being launched from a cannon using Euler's
method. As input quantities you should add the following (Make these
quantities variables so you can easily change values and recalculate):}

\begin{itemize}
\item {\small The initial }$x${\small \ position of the projectile }$x=0.00$

\item {\small The launch angle for the projectile (measured from the positive
}$x${\small \ axis) }$45${\small \ degrees Include a plot of }$y${\small \ vs
}$x${\small \ in your lab notebook. Save your script and record where you
saved it.}
\end{itemize}

\item {\small Copy and modify your script to include air resistance. Air
resistance will add a new resistive force that is proportional to the square
of the projectile's velocity, i.e. } \[
F_{R}=\frac{1}{2}D\rho Av^{2}
\]
{\small where }$D${\small \ is the drag coefficient, }$\rho${\small \ is the
density of the air, }$A${\small \ is the cross-sectional area of the
projectile presented to the air (in our case }$A=\pi r^{2}${\small ), and }$v
${\small \ is the speed. The force is directed opposite the velocity of the
projectile. You will need components of this force. You should convince
yourself that } \begin{align*}
F_{R_{x}}  & =-\frac{1}{2}D\rho Av\left(  n\right)  v_{x}\left(  n\right) \\
F_{R_{y}}  & =-\frac{1}{2}D\rho Av\left(  n\right)  v_{y}\left(  n\right)
\end{align*}
{\small where } \[
v\left(  n\right)  =\sqrt{\left(  v_{x}\left(  n\right)  \right)  ^{2}+\left(
v_{y}\left(  n\right)  \right)  ^{2}}
\]
{\small and }$v_{x}\left(  n\right)  ${\small \ and }$v_{y}\left(  n\right)
${\small \ are the components of the velocity. This form of the components of
the resistive force avoids having to calculate the angle at each of our Euler
steps. As input quantities you should provide the following (Make these
quantities variables so you can easily change values and recalculate):}

\begin{itemize}
\item {\small The mass of the projectile:}
$0.020 \operatorname{kg} $

\item {\small The radius of the projectile:}
$0.02${\small \ meters}

\item {\small The drag coefficient for the projectile:}
$0.2$

\item {\small The density of the air the projectile travels through:}
$1.23 \operatorname{kg} / \operatorname{m} ^{3}$
\end{itemize}

\item {\small Include a scatter plot of your }$x${\small \ and } $y${\small \ positions for each time step in your lab notebook}
\end{enumerate}




\end{document}
